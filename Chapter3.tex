\chapter{Noetherian Rings and Dedekind Rings}

\section{Noetherian rings and modules}

\begin{lemma}
  Let $(T, \leq)$ be a partially ordered set. The following statements are equivalent :
  \begin{enumerate}
    \item Every non-empty subset of $T$ contains a maximal element.
    \item Every increasing sequence $(t_n)_{n\geq 0}$ of elements of $T$ is stationary.
  \end{enumerate}
\end{lemma}
\begin{proof}
  Let $(t_n)$ be an increasing sequence of elements of $T$ with respect to $\leq$ and $t_p$ be a maximal element of $(t_n)$. Then for $n\geq p$, $t_p\leq t_n$, so $t_n = t_p$ for all $n\geq p$.

  Pick $\varnothing \neq S\subseteq P$. Let $x_1\in S$ be arbitrary. Given $x_k\in S$, pick $x_{k+1}\in S$ strictly bigger than $x_k$. By hypothesis, we will eventually run out of bigger elements to pick at say $x_n$. Then by construction there are no larger elements than $x_n$, that is, $x_n$ is a maximal element of $S$.
\end{proof}

\begin{theorem}
  Let $\calr$ be a ring and $\calm$ be an $\calr$-module. The following statements are equivalent.
  \begin{enumerate}
    \item Every non-empty collection of submodules of $\calm$ contains a maximal element.
    \item Every increasing sequence of submodules of $\calm$ is stationary.
    \item Every submodule of $\calm$ is of finite type.
  \end{enumerate}
\end{theorem}

\begin{proof}
  We will first establish the equivalence of $(2)$ and $(3)$.
  Assume $N_1\subseteq N_2\subseteq N_3\subseteq \cdots$ be an increasing sequence of submodules of $\calm$. From $3$, $N:=\cup_{i\geq 0} N_i$ is a finitely generated submodule of $\calm$. Suppose $N$ is generated by $a_1,\ldots, a_k\in N$. For all $i\in \{1,\ldots, k\}$, there is some $j_i \in \N$ such that $a_i\in N_{j_i}$.
  For $j:=\max\{j_1,\ldots, j_k\}$, we have $a_1,\ldots, a_k\in N_j$. Hence $N_j = N$. Therefore, every increasing sequence of submodules of $\calm$ is stationary.

  Suppose every increasing sequence of submodules of $\calm$ is stationary. Let $N$ be a submodules of $\calm$. For the sake of a contradiction, suppose $N$ is not finitely generated. Any finitely generated submodule of $N$ is not equal to $N$. So we can inductively choose a sequnce $a_i\in N\setminus \langle a_1,\ldots, a_{i-1}\rangle$. The chain : $\langle a_1\rangle \subsetneq \langle a_1,a_2\rangle \subsetneq \cdots$ is strictly increasing contradicting $2$. Hence, every submodule of $\calm$ is finitely generated.

  The equivalence of $(1)$ and $(2)$ follows from Lemma $1$.
\end{proof}

\begin{definition}[Noetherian Module]
  An $\calr$-module $\calm$ is called Noetherian if it satisfies the equivalent conditions of Theorem $1$.
\end{definition}
\begin{definition}[Noetherian Ring]
 A ring $\calr$ is called Noetherian if, considered as an $\calr$-module, it is a Noetherian module.
\end{definition}

\begin{prop}
  Let $0\rightarrow \calm'\xrightarrow{f}\calm\xrightarrow{g}\calm''\rightarrow 0$ be an exact sequnce of $\calr$-modules. Then $\calm$ is Noetherian if and only if $\calm'$ and $\calm''$ are Noetherian.
\end{prop}

\begin{proof}
  Suppose $\calm$ is Noetherian. Since $\calm'$ is isomorphic to a sub module of $\calm$, $\calm'$ is Noetherian. Let $N''$ be a submodule of $\calm''$. Then $g^{-1}(N'')$ is a submodule of $\calm$. Therefore there exist $x_1,\ldots, x_r\in g^{-1}(N'')$ such that $g^{-1}(N'')$ is generated by $x_1,\ldots, x_r$. Since $g$ is surjective, we have $N''=g(g^{-1}(N''))$. It follows that $N''$ is generated by $g(x_1),\ldots, g(x_r)$. Thus $M''$ is Noetherian.

  Conversely, suppose $\calm'$ and $\calm''$ are Noetherian. Let $N$ be a submodule of $\calm$. Then $g(N)$ is a submodule of $\calm''$. Therefore, there exist $x_1,\ldots, x_r\in N$ such that $g(x_1),\ldots, g(x_r)$ generate $g(N)$. Next, $f^{-1}(N)$ is a submodule of $\calm'$. Therefore there exist $y_1,\ldots, y_s\in f^{-1}(N)$ such that $f^{-1}(N)$ is
   generated by $y_1,\ldots, y_s$. We claim that $N$ is generated by $x_1,\ldots, x_r, f(y_1),\ldots, f(y_s)$. Let $z\in N$. Then
    $g(z)=\sum_{i=1}^r a_i g(x_i)$ with $a_1,\ldots, a_r\in \calr$. Let $z'=z-\sum_{i=1}^r a_i x_i$. Then $z'\in N\cap \ker{g} = N\cap \im{f}$. Therefore $z'=f(x')$ with $x'\in f^{-1}(N)$. There exist $b_1,\ldots, b_s\in \calr$ such that $x'=\sum_{j=1}^s b_j y_j$. Thus $z=\sum_{i=1}^r a_i x_i + \sum_{j=1}^s b_j f(y_j)$.
\end{proof}

\begin{prop}
  Let $\calr$ be a ring, $\calm$ an $\calr$-module, and $\calm'$ a submodule of $\calm$. Then $\calm$ is Noetherian if and only if $\calm'$ and $\calm/\calm'$ are Noetherian.
\end{prop}

\begin{proof}
  Consider the short exact sequence :
  \[0\rightarrow \calm'\xrightarrow{f}\calm\xrightarrow{g}\calm/\calm'\rightarrow 0\]
  where $f$ is the inclusion map and $g$ takes $x\in \calm$ to $x+ \calm'$. Note that $f$ is clearly injective and $g$ is surjective because for all $x+\calm'\in \calm/\calm'$, $g(x)=x+\calm'$. We observe that $\ker{g}=\{x\in \calm : x+\calm' = \calm'\} = \calm' = \im{f}$.
  Applying proposition $1$ to this sequence completes the proof.
\end{proof}


\begin{corollary}
  Let $\calr$ be a ring and let $\calm_1, \ldots, \calm_n$ be Noetherian $\calr$-modules. Then the $\calr$-module product $\prod_{i=1}^n E_i$ is Noetherian.
\end{corollary}
\begin{proof}
  For $n=2$, we want to show that if $\calm_1$ and $\calm_2$ are Noetherian, then the product $\calm_1\times \calm_2$ is Noetherian. Consider the sequence :
  \[0\rightarrow \calm_1 \xrightarrow{f}\calm_1\times \calm_2 \xrightarrow{g}\calm_2 \rightarrow 0,\]where $f(x) = (x,0)$ for all $x\in \calm_1$ and $g(x,y)=y$ for all $x,y \in \calm_1\times \calm_2$. Note that $f$ is injective and $g$ is surjective. We observe that $\im{f}=\{(x,0):x\in \calm_1\}\cong \calm_1  = \ker{g}$. Therefore, this is a short exact sequence. Applying Proposition $1$ to this sequence proves the result for $n=2$. Inductively, it follows that $\prod_{i=1}^n E_i$ is Noetherian.
\end{proof}

\begin{corollary}
  Let $\calr$ be a Noetherian ring and let $\calm$ be an $\calr$-module of finite type. then $\calm$ is a Noetherian module.
\end{corollary}
\begin{proof}
  Suppose $\calr$ is generated by $x_1,\ldots x_r$. We prove the assertion by induction on $r$. First suppose $r=1$. Let $g: \calr \to \calm$ be the map defined by $g(a)=ax_1$. Then $g$ is a surjective homomorphism and it follows that $\calm$ is Noetherian from Propostion $1$.

  Now, suppose $r\geq 2$. Let $\calm' = Ax_r$. Let $g:\calm \to \calm/\calm'$ be the natural surjection. Then $\calm/\calm'$ is generated by $g(x_1),\ldots, g(x_{r-1})$. Therefore by induction both $\calm'$ and $\calm/\calm'$ are Noetherian. Therefore by Propostion $1$, $\calm$ is Noetherian.
\end{proof}

\section{An application concerning integral elements}

\begin{lemma}
  Let $\calr$ be an integrally closed ring. Let $\mathcal{K}$ be its field its field of fractions, $\mathcal{L}$ be an extension of finite degree $n$ of $\mathcal{K}$, and $\calr'$ is the integral closure of $\calr$ in $\mathcal{L}$. Suppose $\mathcal{K}$ is of characteristic $0$. Then $\calr'$ is an $\calr'$-submodule of a free $\calr$-module of rank $n$.
\end{lemma}

\begin{proof}
  Let $(x_1,\ldots, x_n)$ be a base of $\mathcal{L}$ over $\mathcal{K}$. Each $x_i$ is algebraic over $\mathcal{K}$, so for any $i$, we have an equation of the form $a_n x^n + a_{n-1}x^{n}+\cdots+a_0 = 0 (a_j\in \calr\ \forall\ j)$. We may assume $a_n \neq 0$. Multiplying through by $a_{n}^{n-1}$, we see that $a_n x_i$ is integral over $\calr$.
  Put $x_i' = a_n x_i$. Then $(x_1', \ldots, x_n')$ is a base for $\mathcal{L}$ over $\mathcal{K}$ contained in $\calr'$.
  Hence, there exists another base $(y_1,\ldots, y_n)$ of $\mathcal{L}$ over $\mathcal{K}$ such that $\operatorname{Tr}(x_i' y_j) = \delta_{ij}$. Let $z\in \calr'$. Since $(y_1,\ldots, y_n)$ is a base for $\mathcal{L}$ over $\mathcal{K}$, we may write $z = \sum_{j=1}^n b_j y_j$ with $b_j\in \mathcal{K}$. For any
  $i$, we have $x_i' z\in \calr'$. Therefore, $\operatorname{Tr}(x_i'z)\in \calr$. Thus, $\operatorname{Tr}(x_i' z) = \operatorname{Tr}(\sum_{j}b_j x_i' y_j) = \sum b_j \operatorname{Tr}(x_i' y_j) = \sum b_j \delta_{ij} = b_i$. Hence, it follows that $b_i \in \calr$ for all $i$, which implies that $\calr'$ is a submodule of the free $\calr$-module $\sum_{j=1}^n \calr y_j$.
\end{proof}

\begin{prop}
  Let $\calr$ be a Noetherian integrally closed ring. Let $\calk$ be its field of fractions, $\call$ a finite extension of $\calk$, and $\calr'$ the integral closure of $\calr$ in $\call$. Suppose that $\calk$ is of charactersitic $0$. Then $\calr'$ is a $\calr$-module of finite type and a Noetherian ring.
\end{prop}


\begin{proof}
From previous lemma we know that $\calr'$ is a submodule of a free $\calr$-module of rank $n$. Thus $\calr$' is a $\calr$-module of finite type, and therefoe, a Noetherian module. On the other hand, the ideals of $\calr'$ are special cases of $\calr$-submodules of $\calr$'. They satisfy the maximal condition, so $\calr'$ is a Noetherian ring.
\end{proof}

\section{Some preliminaries concerning ideals}

\begin{definition}[Prime and Maximal Ideals]
  Let $\calr$ be a non-zero ring and $\calp$ be an ideal of $\calr$. We say that $\calp$ is a \textbf{prime ideal} of $\calr$ is the following hold :
  \begin{enumerate}
    \item $\calp \neq \calr$,
    \item if there exist ideals $\mathcal{I},\mathcal{J}$ of $\calr$ such that $\mathcal{I},\mathcal{J}\subseteq \calr$, then $\mathcal{I}\subseteq \calp$ or $\mathcal{J}\subseteq \calp$.
  \end{enumerate}

      An ideal $\calm$ of $\calr$ is called a \textbf{maximal ideal} if the following hold :
      \begin{enumerate}
        \item $\calm \neq \calr$,
        \item if there exists any ideal $\mathcal{I}$ of $\calr$ such that $\calm\subseteq \mathcal{I}$, then either $\mathcal{I}=\calm$ or $\mathcal{I}=\calr$.
      \end{enumerate}
\end{definition}

\begin{prop}
  Let $\calr$ be a non-zero commutative ring with unity. Then an ideal $\calp$ is prime ideal if and only if for any $a,b\in \calr$ whenever $a\cdot b\in \calp$, then $a\in \calp$ or $b\in \calp$.
\end{prop}

\begin{proof}
  Let $x,y\in \calr$ such that $xy \in \calp$. So, $(xy)\subseteq \calp$. Since $\calr$ is commutative $(xy)=(x)(y)\subseteq \calr$. By definition, either $(x)\subseteq \calp$ or $(y)\subseteq \calp$. Hence, either $x\in \calp$ or $y\in \calp$, proving the forward direction.

  Let $\mathcal{I},\mathcal{J}\subseteq \mathcal{P}$. Without loss of generality, let $\mathcal{I}\subsetneq \calp$. Then there exists $x\in \cal{I}$ such that $x\notin \calp$. Let $y\in \mathcal{J}$. Since $\mathcal{J}$ is an ideal, $xy\in \mathcal{J}$. But $xy \in \mathcal{I}\mathcal{J}\subseteq \calp$,
  so $xy\in \calp$. Hence, either $x\in \calp$ or $y\in \calp$. Since we assumed $x\notin \calp$, $y\in \calp$. Since choice of $y$ was arbitrary, $\mathcal{J}\subseteq \calp$, proving the reverse direction.
\end{proof}

\begin{prop}
  Let $\calr$ be a commutative ring with unity. Then an ideal $\calp$ of $\calr$ is a prime ideal if and only if the quotient ring $\calr/\calp$ is an integral domain.
\end{prop}
\begin{proof}
  Let $\calp$ be a prime ideal, $a+\calp, b+\calp \in \calr/\calp$. If $(a+\calp)(b+\calp)=p$, then we get $ab \in \calp$. Hence, $a\in \calp$ or $b\in \calp$. So $a+\calp = \calp$ or $b+\calp = \calp$. Hence, $\calr/\calp$ is an integral domain.

  Now let $\calr/\calp$ be an integral domain. Let $a\notin \calp$ and $b\notin \calp$ then $a+\calp = \calp$ and $b+\calp \neq \calp$. Therefore $(a+\calp)(b+\calp)\neq p$. Hence, $ab+\calp \neq \calp$ implies $ab\notin \calp$. Hence, $\calp$ is prime.
\end{proof}

\begin{prop}
  Let $\calr$ be a commutative ring with unity. Then an ideal $\calm$ of $\calr$ is a maximal ideal if and only if the quotient ring $\calr/\calm$ is a field.
\end{prop}

\begin{proof}
  Let $\calm$ be a maximal ideal. Let $a+\calm$ be a non zero element of $\calr/\calm$. Hence $a+\calm \neq \calm$, or $a\notin \calm$. Consider the ideal $\calm + (a)$. Observe that $\calm \subsetneq \calm + (a)\subseteq \calr$. Since $\calm$ is maximal $\calm + (a)=\calr$. Hence, there exist $r\in \calr$, $m_0\in \calm$ such that $m_0 + ra = 1$.
  It follows that $(a+\calm)(r+\calm) = ar + \calm = 1-m_0 + \calm = 1 + \calm$. Hence, $a+\calm$ is a unit. Since choice of $a$ was arbitrary, it follows every non-zero element in $\calr/\calm$ is a unit. Hence, $\calr/\calm$ is a field.

  Now let $\mathcal{I}$ be an ideal such that $\calm\subsetneq \mathcal{I}\subseteq \calr$. Then there exists an $r\in \mathcal{I}\setminus \calm$. Since $\calr/\calm$ is a field, since $r\notin \calm$, $r+\calm$ has an inverse, say $r_1 + \calm$. Now $(r+\calm)(r_1 + \calm) = 1 + \calm \implies rr_1 + \calm = 1 + \calm$,
  or $rr_1 - 1 \in \calm \subset \mathcal{I}$. Since $r\in \mathcal{I}, r_1\in \calr$, we get $rr_1 \in \mathcal{I}$, so $rr_1 - (rr_1 -1)\in \mathcal{I}$ or $1\in \mathcal{I}$. Hence, $\mathcal{I} = \calr$. Therefore, $\calm$ is maximal.
\end{proof}

\begin{lemma}
  Let $\calr$ be a ring, $\calp$ be a prime ideal of $\calr$, and let $\calr'$ be a subring of $\calr$.  Then $p\cap \calr'$ is a prime ideal of $\calr'$.
\end{lemma}

\begin{proof}
  Let $x\in \calr'$ and $\alpha \in \calp\cap \calr'$, then $\alpha \in \calp \implies x\alpha \in \calp$. Since, $x\in \calr'$ and $\alpha \in \calr$, we ger $x\alpha \in \calp\cap \calr'$. Therefore,
  $p\cap \calr'$ is an ideal of $\calr'$. Consider the map $\psi$ defined as $\psi : \calr'/\calr'\cap \calp \to \calr/\calp, x+ \calr' \cap \calp \mapsto x + \calp$. $\psi$ is clearly a homomorphism. The kernel of $\psi = \{x+\calr'\cap \calp : x+\calp = \calp\} =
  \{x+\calr'\cap \calp : x\in \calp\} = \{\calr'\cap \calp\}$. Hence, we have $\calr'/\calr'\cap \calp$ is a subring of $\calr/\calp$, so it must be an integral domain.
\end{proof}


\begin{definition}[Sum and product of ideals]
  Let $\calr$ be a ring and $\mathcal{I},\mathcal{J}$ be two ideals of $\calr$. We define the sum of two ideals $\mathcal{I},\mathcal{J}$ as follows :
  \[\mathcal{I}+\mathcal{J} := \{x+y : x\in \mathcal{I}, y\in \mathcal{J}\}.\]
  We define the product of two ideals $\mathcal{I},\mathcal{J}$ of $\calr$ as follows :
  \[\mathcal{I}\mathcal{J} := \left\{\sum_{i=1}^n x_i y_i : n\in \mathbb{N}, x_i \in \mathcal{I}, y_i\in \mathcal{J}\right\}.\]
\end{definition}

\begin{lemma}
  If a prime ideal $\calp$ of $\calr$ contains a product $\mathcal{I}_1\cdots \mathcal{I}_n$ of ideals. Then $\calp$ contains at least one of the ideals $\mathcal{I}_i$.
\end{lemma}

\begin{proof}
  If $\mathcal{I}_i\not\subset \calp$ for any $i$, then there exist $a_i \in \mathcal{I}_i \setminus \calp$ for all $i$. Therefore, $a_i\cdots a_n \notin \calp$, since $\calp$ is prime. But $a_i \cdots a_n \in \mathcal{I}_1\cdots \mathcal{I}_n$ which contradicts the hypothesis of the lemma.
\end{proof}

\begin{lemma}
  In a Noetherian ring every ideal contain a product of prime ideals. In a Noetherian integral domain $\calr$, every non-zero ideal contains a product of prime ideals.
\end{lemma}

\begin{proof}
  Let $\Phi$ be the set of non zero ideals of $\calr$ which don't contain product of non-zero prime ideals. We want to show that $\Phi$ is non-empty. For the sake of a contradiction, let $|\Phi|>0$. Since $\calr$ is Noetherian, $\Phi$ contains a maximal element $\mathcal{B}$. The ideal $\mathcal{B}$ cannot be prime; otherwise $\mathcal{B}\in \Phi$. Thus, there exist $x,y\in \calr\setminus\mathcal{B}$ such that $xy\in \mathcal{B}$. The ideals $\mathcal{B}+(x)$ and
   $\mathcal{B}+(y)$ contain $\mathcal{B}$ as a proper subset. Therefore, since $\mathcal{B}$ is maximal, they do not belong to $\Phi$. It follows that they both contain products of non zero prime ideals.
   \[\mathcal{B} + (x)\supset p_1\cdots p_n, \quad \mathcal{B}+(y)\supset q_1\cdots q_r\]
   Since $xy\in \mathcal{B}$,
   \[(\mathcal{B}+(x))(\mathcal{B}+(y))\subset \mathcal{B}.\]
   Hence, $p_1\cdots p_n\cdot q_1\cdots q_r \subset \mathcal{B}$, a contradiction. Hence, $|\Phi|=0$.
\end{proof}


\begin{definition}[Fractional ideals]
  Let $\calr$ be an integral domain and $\mathcal{K}$ be its field of fractions. Let $\mathcal{I}$ be an $\calr$-submodule. We call $\mathcal{I}$, a fractional ideal of $\mathcal{K}$ if there exists a $d\in \calr\setminus\{0\}$ such that $d\cdot \mathcal{I}\subseteq \calr$.
\end{definition}
  The ordinary ideals of $\calr$ are fractional ideals with $d=1$. They are also called \textbf{integral ideals} to distinguish them from fractional ideals.

\begin{prop}
  The following are true :
  \begin{enumerate}
    \item Any $\calr$-submodule $\mathcal{I}$ of finite type contained in $\mathcal{K}$ is a fractional ideal.

    \item If $\calr$ is Noetherian, every fractional ideal $\mathcal{I}$ is an $\calr$-module of finite type.

    \item If $\mathcal{I}$ and $\mathcal{I}'$ are fractional ideals, then the sets $\mathcal{I}\cap \mathcal{I}'$, $\mathcal{I}+\mathcal{I}'$, and $\mathcal{I}\mathcal{I}'$ are all fractional ideals.
  \end{enumerate}
\end{prop}

\begin{proof}
  \begin{enumerate}
    \item Since $\mathcal{I}$ is an $\mathcal{R}$-submodule of finite type it must be generated by a finite set of generators $\langle a_1, \ldots, a_n\rangle$. If $a_i = p_i/q_i$  for all $i$, then the product $d = \prod_{i=1}^n q_i$ is a common denominator for $\mathcal{I}$.

    \item Since $d\cdot \mathcal{I} \subseteq \calr$, we get $\mathcal{I} \subseteq d^{-1}\calr$. Since $d^{-1}\calr$ is isomorphic to $\calr$, $\mathcal{I}$ is a Noetherian module.

    \item If $d$ and $d'$ are the common denominators for $\mathcal{I}$ and $\mathcal{I}'$ respectively then $dd'$ is a common denominator for $\mathcal{I}\cap \mathcal{I}'$, $\mathcal{I}+\mathcal{I}'$, and $\mathcal{I}\mathcal{I}'$.
  \end{enumerate}
\end{proof}

\section{Dedekind Domains}

\begin{definition}[Dedekind domain]
  An integral domain $\calr$ is called a Dedekind domain if it is Noetherian and integrally closed, and if every non-zero prime ideal of $\calr$ is maximal.
\end{definition}

\begin{eg}
  Every principal ideal ring is a Dedekind domain.
\end{eg}

\begin{theorem}
  Let $\calr$ be a Dedekind domain, $\calk$ be its field of fractions. Let $\call$ be a finite extension of $\calk$ and $\calr'$ be the integral closure of $\calr$ in $\call$. If $\calk$ is of characteristic $0$. Then $\calr'$ is a Dedekind domain and an $\calr$-module of finite type.
\end{theorem}

\begin{proof}
  We need to show three things. That $\calr'$ is integrally close, that $\calr'$ is Noetherian, and that every non-zero prime ideal of $\calr$ is maximal. The first part is done for us by construction. From Proposition $3$, we get that $\calr'$ is Noetherian and a $\calr$-module of finite type. It remains to show that every prime ideal $\calp'\neq (0)$ of $\calr'$ is maximal. Let $x\in \calp'\notin (0)$ and the following be its minimal polynomial over $\calr$ :
  \[x^n + a_{n-1}x^{n-1}+\cdots+a_1 x + a_0 = 0\tag{$a_i\in \calr$}.\]
  Note that $a_0\neq 0$, because if not then dividing through by $x$, we get a polynomial of lower degree. Note that since $a_0 = -x(x^{n-1}+a_{n-1}x^{n-2}+\cdots + a_1)$, we get that $a_0 \in \calr' x$. But since $a_0 \in \calr$, we have that $x\in \calr' x \cap \calr \subseteq \calp'\cap \calr$. Hence, $\calp'\cap \calr \neq (0)$.
  Since $\calp'$ is a prime ideal, $\calp'\cap \calr$ is a prime ideal. Since $\calr$ is a Dedekind ring $\calp'\cap \calr$ is a maximal ideal of $\calr$ and so $\calr/\calp'\cap\calr$ is a field.
  Consider the map $\varphi : \calr \to \calr'/\calp'$ such that $x\mapsto x+\calp'$. This is clearly a well defined homomorphism. The kernel of this map is $\ker{\varphi}:=\{x\in \calr : x+\calp' = \calp'\} = \calr \cap \calp'$. It follows that $\calr/\calr\cap \calp'$ is a subring of $\calr'/\calp'$.


  We note that $\calr'/\calp'$ is integral over $\calr/\calp'\cap \calr$
  \footnote{Pick an element $x+\calp' \in \calr'/\calp'$. Since $x\in \calr'$, there exist $n\in \mathbb{Z}, a_0,\ldots, a_{n-1} \in \calr$ such that $a_0 + a_1 x + \cdots + x^n = 0$. Hence, $(a_0 + \calr\cap \calp') + (a_1+\calr\cap \calp') (x+\calp')+\cdots + (1+\calr\cap \calp')(x+\calp')^n = \calr\cap \calp'$ or
  $x+\calp'$ is integral over $\calr/\calr\cap \calp'$.}. Thus $\calr'/\calp'$ is a field, so $\calp'$ is maximal.
\end{proof}


\begin{theorem}
  Let $\calr$ be a Dedekind domain which is not a field. Every maximal ideal of $\calr$ is invertible in the monoid of fractional ideals of $\calr$.
\end{theorem}
\begin{proof}
  The set of frational ideals forms a monoid under multiplication. Closure follows Proposition $7.3$. Associativity follows from the associativity of $\calr$, and $\calr$ acts as the identity.

  Let $\calm$ be a maximal ideal of $\calr$. Then $\calm \neq (0)$, since $\calr$ is not a field. Put
  \[\calm' = \{x\in \calk | x\calm \subseteq \calr\}.\]

  Note that $\calm'$ is an $\calr$-submodule of $\calk$ and is a fractional ideal of $\calr$. We need to show that $\calm\calm' = \calr$. From the definition of $\calm'$, it must be that $\calm\calm'\subseteq \calr$. As $\calm$ is a maximal ideal, $\calm = \calr \calm \subseteq \calm'\calm \subseteq \calr$
  we get that either $\calm \calm' = \calm$ or $\calm\calm' = \calr$. It suffices to show that $\calm\calm' \neq \calm$.

  For the sake of contradiction, suppose $\calm'\calm = \calm$. Then for any $x\in \calm'$ we have $x\calm \subseteq \calm$, $x^2\calm \subseteq x\calm \subseteq \calm$. Inductively, $x^n \calm \subseteq \calm$ for all $n\in \N$. Hence any non-zero element $d\in \calm$ acts as a common denominator for all powers $x^n$ of $x, n\in \N$. It follows that $\calr[x]$
  is a fractional ideal of $\calm$. Since $\calr$ is Noetherian, $\calr[x]$ is a $\calr$-module of finite type, so $x$ is integral over $\calr$. But $\calr$ is integrally closed, therefore $x\in\calr$ and hence $\calm'\calm = \calm \implies \calm' = \calr$. It suffices to show that $\calm'$ is never equal to $\calr$.

  Let $0\neq a \in \calm$. The ideal $\calr a$ contains a product $p_1 p_2\cdots p_n$ of non-zero prime ideals. Let $n$ be as small as possible. Note that $\calm \supseteq \calr a \supset p_1 p_2\cdots p_n$, so there exists $i\in \{1,\cdots, n\}$ such that $\calm \supseteq p_i$. Since $p_i$ is prime, it is also maximal ($\calr$ is a Dedekind domain) we get that $\calm = p_i$.
  Therefore, $\calr a \supseteq p_i \prod_{j\neq i}p_j$ and $\calr \supsetneq \prod_{j\neq i}p_j$, since our $n$ was minimal. Hence, there exists a $b\in \prod_{j\neq i}p_j$ such that $b\notin \calr a$. But
  $\calm \prod_{j\neq i}p_j \subseteq \calr a$ so $\calm b \subseteq \calr a$ or $ \calm ba^{-1}\subseteq \calr$. From the definition of $\calm'$, it follows that $ba^{-1}\in \calm'$. Since $b\notin \calr a$, we get $ba^{-1}\notin \calr$. Therefore $\calm' \neq \calr$.

\end{proof}


\begin{theorem}
  Let $\calr$ be a Dedekind domain and let $\operatorname{spec}(\calr)$ be the set of non-zero prime ideals of $\calr$. Then :
  \begin{enumerate}
    \item Every non-zero fractional ideal $\mathfrak{b}$ of $\calr$ may be uniquely expressed in the form :
    \[\mathfrak{b} = \prod_{\frakp\in \operatorname{spec}(\calr)} {\frakp}^{n_p(\mathfrak{b})},\]
    where, for any $\frakp\in \operatorname{spec}(\calr),n_{\frakp}(\mathfrak{b})\in \Z$ and for almost all $\frakp\in \operatorname{spec}(\calr), n_{\frakp}(\mathfrak{b}) = 0$.
    \item The monoid of non-zero fractional ideals of $\calr$ is a group.
  \end{enumerate}
\end{theorem}

\begin{proof}
  Let $\mathfrak{b}$ be a non-zero fractional ideal of $\calr$. Then by definition there exists a $d\in \calr\setminus \{0\}$ such that $d\mathfrak{b}\subseteq \calr$, or $d\mathfrak{b}$ is an integral ideal of $\calr$. Let $\Gamma$ be the set of non-zero ideals in $\calr$ which are not product of prime ideals. For the sake of contradiction, lets assume $|\Gamma|>0$. By Zorn's Lemma, there exists $\fraka$ be a maximal element of $\Gamma$. Since $\calr$ is the product of the empty collection of prime ideals, so $\fraka = \calr$.

   Every ideal is contained in a maximal ideal, so let $\fraka \subseteq \frakp$. Let $\frakp'$ be the inverse fractional ideal of $\frakp$ in the monoid of fractional ideals of $\frakr$, the existence of which we proved earlier. Now since $\fraka \subseteq \frakp$, we get $\fraka\frakp' \subseteq \frakp\frakp' = \calr$. Since $\calr \subseteq \frakp'$, $\fraka \subseteq \fraka\frakp'$; in fact $\fraka\frakp'  \neq \fraka$ (if
    $\fraka\frakp' = \fraka$ and if $x\in \frakp'$, then $x\fraka \subseteq \fraka$, $x^n\fraka\subseteq \fraka$ for all $n,x$ integral over $\frakr$, and $x\in \frakr$. But this is impossible, since $\frakp' \neq \frakr$ (otherwise $\frakp' = A$ and $\frakp\frakp'=\frakp$).) According to the maximality of $\fraka$ in $\Gamma$, we have
     $\fraka\frakp'\notin \Gamma$, so $\fraka\frakp' = \frakp_1\cdots\frakp_n$,  a product of prime ideals. Multiplying by $\frakp$, we see that $\fraka = \frakp\cdot\frakp_1\cdots \frakp_n$. Thus every integral ideal of $\calr$ is a product of prime ideals.


  Consider the uniqueness. Suppose that :
  \[\prod_{\frakp \in \operatorname{spec}(\calr)} \frakp^{n(\frakp)} = \prod_{\frakp \in \operatorname{spec}(\calr)} \frakp^{m(\frakp)} \implies \prod_{\frakp \in \operatorname{spec}(\calr)} \frakp^{n(\frakp)- m(\frakp)} = \calr.\]

  If $n(\frakp)-m(\frakp)\neq 0$ fo some ideals $\frakp \in \operatorname{spec}(\calr)$, we may separate the positive and negative exponents and write :
  \[\frakp_1^{\alpha_1}\cdots \frakp_r^{\alpha_r}= \frakq_1^{\beta_1}\cdots\frakq_s^{\beta_s},\]
  where $\frakp_i, \frakq_j\in \operatorname{spec}(\calr), \alpha_i, \beta_j >0, \frakp_i \neq \frakq_j$ for all $i$ and $j$. Thus $\frakp_1$ contains $ \frakq_1^{\beta_1}\cdots\frakq_s^{\beta_s}$; $\frakp_1 \supset \frakq_j$,
   for some $j$, say $\frakp_1 \supset \frakq_1$. But $\frakp_1$ and $\frakq_1$ are both maximal, which implies $\frakp_1=\frakq_1$, which is a contradiction.

   We now note that $\prod_{\frakp \in \operatorname{spec}(\calr)}\frakp^{n_\frakp(\frakb)}$ is the inverse of $\frakb$. Hence, the monoid of non-zero fractional ideals of $\calr$ is a group.
\end{proof}
