\chapter{Ideal Classes and the Unit Theorem}

\section{Preliminaries concerning discrete subgroups of $\R^n$}

\begin{definition}[Discrete Subgroup of $\R^n$]
A subgroup $H$ of $\R^n$ is discrete if and only if, for any compact subset $K$ of $\R^n$, the intersection $H\cap K$ is finite.
\end{definition}

\begin{theorem}
  Let $H$ be a discrete subgroup of $\R^n$. Then $H$ is generated over $\Z$ by $r$ vectors $e_1,\ldots, e_r$ which are linearly independent over $\R$.
\end{theorem}

\begin{proof}
  Choose $\mathbf{e}=e_1,\ldots, e_r$ in $H$ such that they are $\R$-linearly independent and $r$ is maximal. Define $\mathscr{P}$ as follows :
  \[\mathscr{P} := \left\{\sum_{i=1}^r \alpha_i e_i | \alpha_i \in [0,1]\right\},\]

  The set $\mathscr{P}$ is called the fundamental pallelogram of $H$ with respect to the basis $e_1, \ldots, e_r$. We can immediately see that $\mathscr{P}$ is compact because it is homeomorphic to $\R^n$.
  Let $x\in H$. Then we can write $x$ in the form :
  \[x = \sum_{i=1}^r \lambda_i e_i,\]
  for $\lambda_i \in \R$. If $x$ cannot be written in this form, we can add $x$ to $\{e_1,\ldots, e_r\}$, contradicting the maximality of $r$.
  Now for $j\in \Z$, let :
  \[x_j := jx - \sum_{i=1}^r \lfloor \lambda_i j\rfloor e_j = \sum_{i=1}^r (\lambda_i j - \lfloor\lambda_i j\rfloor)e_i \tag{$j\in \Z$}.\]
  Hence, $x_j\in \mathscr{P}$. It follows that $x_j \in \mathscr{P}\cap H$.
  Since $H$ is discrete, $\mathscr{P}\cap H$ is finite since $\mathscr{P}$ is compact. For $j=1$, we have :
  \[x = x_1 + \sum_{i=1}^r \lfloor \lambda_i \rfloor e_i.\]
  So $x$ is in the $\Z$ span of a finite set. Hence, $H$ is finitely generated over $\Z$. Since $\mathscr{P}\cap H$ is finite, there exist $j\neq k$ such that $x_j = x_k$. It follows that :
  \[\sum_{i=1}^r \lambda_i (j-k)e_i = \sum_{i=1}^r (\lfloor j\lambda_i\rfloor - \lfloor k\lambda_i\rfloor)e_i.\]
  Linear independence of $\{e_i\}$ gives us :
  \[\lambda_i (j-k)=\lfloor j\lambda_i\rfloor - \lfloor k\lambda_i\rfloor.\]
  Hence $\lambda_i \in \Q$ for all $i$. So far we've shown that for any $x\in \mathscr{P}\cap H$, $x=\sum_{i=1}^r \lambda_i e_i$ where $\lambda_i\in \Q$ for all $i$. Let $d$ be the least common multiple of the denominators of $\lambda_i$'s. Then for any $x\in H$, we know :
  \[x = x_1+\sum_{i=1}^r \lfloor \lambda_i \rfloor e_i \in \frac{1}{d}\sum_{i=1}^r \Z e_i.\] We conclude that :
  $H \subseteq \frac{1}{d}\sum_{i=1}^r \Z e_i \implies \sum_{i=1}^r \Z e_i \subseteq H \subseteq \frac{1}{d}\Z e_i$. Therefore we must have that $H$ is finitely generated of rank $r$ over $\Z$ on some linear combination of the vectors $\{\frac{1}{d}e_i\}$. This basis is linearly independent over $\R$ as desired.
\end{proof}

\begin{definition}[Lattice]
  A discrete subgroup of rank $n$ of $\R^n$ is called a lattice in $\R^n$.
\end{definition}

By Theorem $1$, a lattice is generated over $\Z$ by a base of $\R^n$, wich is then a $\Z$-base for the given lattic. For each $\Z$-base $e = (e_1,\ldots, e_n)$ of a lattice $H$ we shall write $\mathscr{P}_e$ for the half open parallelotope :
\[\mathscr{P}_e = \left\{x\in \R^n | x = \sum_{i=1}^n \alpha_i e_i, \alpha_i \in [0,1)\right\}\]

Thus every point of $\R^n$ is congruent modulo $H$ to one and only one point of $P_e$ for any fixed $e$ (we say, in this case, that $P_e$ is a fundamental domain for $H$). We shall write $\mu$ to denote the Lebesgue measure in $\R^n$, i.e. if $S$ is a measurable subset of $\R^n$, $\mu(S)$ will stand for its measure (which we will also call its volume).


\begin{lemma}
  The volume $\mu(\scrap_e)$ is independent of the base $e$ chosen for $H$.
\end{lemma}

\begin{proof}
  Let $f = (f_1,\ldots, f_n)$ be another base for $H$. Then :
\[f_i = \sum_{j=1}^n \alpha_ij e_j \tag{$\alpha_{ij}\in \Z$}.\]
By calculus we know that $\mu(\scrap_f)=|\det({\alpha_{ij}})|\mu(\scrap_e)$. The change of bases matrix $(\alpha_{ij}) \in GL_n(\Z)$, so $\det{(\alpha_{ij})}=\pm 1$. Hence, $\mu(\scrap_f)=\mu(\scrap_e)$.
\end{proof}

\begin{definition}[Volume of a Lattice]
  The volume of the parallelotope $\scrap_e$ associated with any base $e$ of $H$ is called the volume of the lattice $H$ and is denoted by $\vol(H)$.
\end{definition}

\begin{theorem}[Minkowski]
  Let $H$ be a lattice in $\R^n$ and let $S$ be a measurable subset of $\R^n$ such that $\mu(S)> \vol(H)$. Then there exist two distinct points $x,y\in S$ such that $x-y\in H$.
\end{theorem}

\begin{proof}
  Consider the sets $S_x = S\cap (x+\scrap_e)$, where $x\in H$. Notice that these sets form a partition of $S$, i.e. they are pairwise disjoint and :
  \[S = \cup_{x\in H} S_x.\]
  In particular we have :
  \[\vol(S) = \sum_{x\in H}\vol(S_x).\]
  Notice that the translated sets $S_x - x = (S-x)\cap \scrap_e$ are all contained in $\scrap_e$. We want to prove that the $S_x$ cannot be all mutually disjoint. Since $\vol(S_x) = \vol(S_x - x)$, we have :
  \[\vol(H)<\vol(S) = \sum_{x\in H}\vol(S_x) = \sum_{x\in H} \vol(S_x - x).\]
  The facts that $S_x - x \subseteq \scrap_e$ and $\sum_{x\in H}\vol(S_x-x)> \vol(H)$ imply that these sets cannot be disjoint, i.e. there exist two distinct vectors $x\neq y \in H$ such that $(S_x - x)\cap (S_y - y)\neq 0$. Let $z$ be any vector in the (non-empty) intersection $(S_x-x)\cap (S_y - y)$ and define :
  \begin{align*}
    z_1 &= z + x \in S_x \subseteq S \\
    z_2 &= z + y \in S_y \subseteq S.
  \end{align*}
  These two vectors satisfy $z_1 - z_2 = x-y \in H$.
\end{proof}

\begin{theorem}[Minkowski's convex body theorem]
  Let $H$ be a full-dimensional lattice in $\R^n$ and let $C\subseteq \R^n$ be a convex set symmetric about the origin $(i.e. x\in C \implies -x\in C)$. Suppose that either :
  \begin{enumerate}
    \item $\vol(C) > \cdot 2^n \cdot \vol(H)$, or
    \item $\vol(C) \geq \cdot 2^n \cdot \vol(H)$ and $C$ is compact.
  \end{enumerate}
  Then $ \C\cap (H\setminus\{0\})\neq \varnothing$.
\end{theorem}
\begin{proof}
  It is easy to see that the volume of the set $\frac12 C = \{x/2 : x\in C\}$ is $2^{-m}\vol(C)$, and therefore, we can apply previous theorem to find $\frac12 x_0,\frac12 x_1\in \frac12 C$ such that $z = \frac12 x_1 - \frac12 x_0 \in H$. Clearly $z = \frac12 x_1 + \frac12 (-x_0) \in C$, since $C$ is convex and symmetric.
\end{proof}


\section{The canonical imbedding of a number field}

\begin{definition}[Canonical imbedding of a number field]
  Let $K$ be a number field and let $n$ be its degree. There are $n$ distinct isomorphisms $\sigma_i : K \to C$. There are exactly $n$, because the minimal polynomial for a primitive element of $K$ ove $\Q$ has only $n$ roots in $\C$. Let $\alpha : \C \to \C$ be complex conjugation. Then, for any $i = 1, \ldots, n$, we have : $\alpha \cird \sigma_i = \sigma_j$ if and only if $\sigma_i(K)\subseteq \R$. We write $r_1$ for the number of indices such that
  $\sigma_i(K)\subseteq \R$. Then $n-r$ is an even number, so we may write :
  \[r_1 + 2 r_2 = n\]
  Let us renumber the $\sigma_i$'s so that $\sigma_i(K)\subseteq \R$ for $1\leq i\leq r_1$ and so that $\sigma_{i+r_2}(x)=\overline{\sigma_j(x)}$ for $r_1 + 1 \leq j \leq r_1 + r_2$. Then the first $r_1+r_2$ isomorphisms determine the last $r_2$. For $x\in K$, we define :
  \[\sigma(x)=(\sigma_1(x), \ldots, \sigma(x_{r_1+r_2}))\in \R^{r_1}\times \C^{r_2}.\]
  We call $\sigma$ the canonical imbedding of $K$ in $\R^{r_1}\times \C^{r_2}$; it is an injective ring homomorphism. We shall frequently identify $\R^{r_1}\times \C^{r_2}$ with $\R^{n}$.
\end{definition}

\begin{prop}
  If $M$ is a free $\Z$-submodule of $K$ of rank $n$ and if $(x_i)_{1\leq i\leq n}$ is a $\Z$-base for $M$ then $\sigma(M)$ is a lattice in $\R^n$, whose volume is :
  \[\vol(\sigma(M)) = 2^{-r_2}|\det_{1\leq i,j\leq n}(\sigma_i(x_j))|.\]
\end{prop}

\begin{proof}
  For fixed $i$ the coordinates of $\sigma(x_i)$ with respect to the canonical base of $\R^n$ are :
  \small{\[\langle\sigma_1(x_i),\ldots,\sigma_{r_1}(x_i), \Re(\sigma_{r_1+1}(x_i)), \Im(\sigma_{r_1+1}(x_i)),\ldots, \Re(\sigma_{r_1+r_2}(x_i)), \Im(\sigma_{r_1+r_2}(x_i))\rangle\]}
  We calculate the determinant $D$ of the matrix whose $i$th column is given as above. We know that $\Re(z) = \frac12(z+\bar{z})$ and $\Im(z)=\frac{1}{2i}(z-\bar{z})$ for $z\in \C$. We obtain $D = (2i)^{-r_2}\det(\sigma_j(x_i))$. We apply the transformation $R_{i}\mapsto iR_{i+1}$ for $i=r_1, r_1+2,\ldots, r_1+2r_n$.
   So we end up with the determinant $D = (2i)^{-r_2}\det_{1\leq i,j\leq n}(\sigma_j(x_i))$. Since $x_i$'s form a base for $K$ over $\Q$, $\det_{1\leq i,j\leq n}(\sigma_j(x_i))\neq 0$ and therefore $D\neq 0$. Thus the vectors $\sigma(x_i)$ are linearly independent in $\R^n$, so that the $\Z$-module they generate (call it $\sigma(M)$) is a lattice in $\R^n$. So we get
   \[\vol(\sigma(M)) = |(2i)^{-r_2}\det(\sigma_j(x_i))| = 2^{-r_2}|\det(\sigma_j(x_i))|\]
   as required.
\end{proof}


\begin{prop}
  Let $d$ be the absolute discriminant of $K$, let $A$ be the ring of integers in $K$, and let $\fraka$ be a non-zero integral ideal of $A$. Then $\sigma(A)$ and $\sigma(\fraka)$ are lattices. Moreover,
  \[\vol(\sigma(A)) = 2^{-r_2}|d|^{\frac12}\quad\text{and}\quad \vol(\sigma(\fraka)) = 2^{-r_2}|d|^{\frac12}N(\fraka).\]
\end{prop}

\begin{proof}
  We know that $A$ and $\fraka$ are free $\Z$-modules of rank $n$, so we may apply the previous proposition. On the other hand, if $(x_i)$ is a $\Z$-base for $A$, then $d = \det_{1\leq i,j\leq n}(\sigma_{i}(x_j))^2$. This proves the first result. The second formula follows from the first and the observation that $\sigma(\fraka)$ is a subgroup of $\sigma(A)$ of index $N(\fraka)$. A fundamental domain for $\sigma(\fraka)$ may obviously constructed as the disjoint union of $N(\fraka)$ copies of a fundamental domain for $\sigma(A)$.
\end{proof}

\section{Finiteness of the ideal class group}
\begin{prop}
  Let $r_1,r_2\in \N$ such that $n=r_1+2r_2$, $t\in \R$ and let $B(r_1,r_2,t)$ be the set of all elements $(y_1,\ldots, y_{r_1}, z_1,\ldots, z_{r_2})\in \R^{r_1}\times \C^{r_2}$ such that :
    \[\sum_{i=1}^{r_1}|y_i| + 2\sum_{j=1}^{r_2}|z_j| \leq t.\]
  Let $\mu$ denote the Lebesgue measure in $\R^n$. Then,
  \[\mu(B(r_1,r_2,t)) = 2^{r_1}\left(\frac{\pi}{2}\right)^{r_2}\frac{t^n}{n!}\tag{for any $t\geq 0$.}\]
\end{prop}

\begin{proof}
  We induct on $n$. The two base cases : $r_1 = 1, r_2 = 0$ and $r_1 = 0$ and $r_2 = 1$. In the former $B(1,0,t) = \{x\in \R : |x|\leq t\}$ has volume $2t = \frac{2^1}{1!}\left(\frac{\pi}{2}\right)^0 t^1$. In the latter case,
  $B(0,1,t) = \{y\in \C : 2 |y| \leq t\}$ which has volume $\pi (t/2)^2 = \frac{2^0}{2!}\left(\frac{\pi}{2}\right)^1 t^2$.

  To go from $n-1\to n$, we could either fix $r_2$ and increment $r_1$ or we could fix $r_1$ and increment $r_2$. In the both cases, we assume the formula is true for $n-1 = r_1 + 2r_2$. Now for $n$, the volume in the first case $r_1 \mapsto r_1 + 1$ fixing $r_2$ is :

  \begin{align*}
    \mu(B(r_1+1,r_2, t)) &= \displaystyle\int_{-t}^t B(r_1, r_2, t-|x|)\mathrm{d}x \\
    &= \displaystyle\int_{-t}^0 B(r_1, r_2, t+x)\mathrm{d}x + \displaystyle\int_{0}^t B(r_1, r_2, t-x)\mathrm{d}x \\
    &= 2^{r_1}\left(\frac{\pi}{2}\right)^{r_2}\frac{1}{(n-1)!}\left[\displaystyle\int_{-t}^0 (t+x)^{n-1} \mathrm{d}x + \displaystyle\int_{0}^t(t-x)^{n-1} \mathrm{d}x\right] \\
    &= 2^{r_1+1}\left(\frac{\pi}{2}\right)^{r_2}\frac{t^n}{n!}
  \end{align*}

  \noindent The volume in the second case $r_2 \mapsto r_2 + 1$ fixing $r_1$ is :
  \begin{align*}
    \mu(B(r_1, r_2+1, t)) &= \displaystyle\int_{\{0\leq |z|\leq t/2\}} B(r_1, r_2, t-2|x|)\mathrm{d}x \\
    &= \frac{2^{r_1}}{(n-2)!}\left(\frac{\pi}{2}\right)^{r_2}\displaystyle\int_{0}^{t/2}\int_{0}^{2\pi}x(t-2x)^{n-1}\mathrm{d}\theta\mathrm{d}x \\
    &= \frac{2^{r_1}}{(n-2)!}\left(\frac{\pi}{2}\right)^{r_2}\cdot 2\pi \displaystyle\int_{0}^{t/2}x(t-2x)^{n-2}\mathrm{d}x \\
    &= \frac{2^{r_1}}{(n-2)!}\left(\frac{\pi}{2}\right)^{r_2}\cdot 2\pi  \dfrac{t^n}{n(n-1)} = 2^{r_1}\left(\frac{\pi}{2}\right)^{r_2+1}\frac{t^n}{n!}
  \end{align*}
  Therefore the formula holds for all $n$.
\end{proof}
















% \begin{prop}
%   Let $K$ be a number field, $n$ its degree, $r_1$ and $r_2$ are integers defined earlier, $d$ the absolute discriminant of $K$, and $\fraka$ a non-zero integral ideal of $K$. Then $\fraka$ contains a non-zero element $x$ such that :
%   \[|N_{K/\Q}(x)|\leq \left(\frac{4}{\pi}\right)^{r_2}\frac{n!}{n^n}|d|^{\frac12}N(\fraka).\]
% \end{prop}
% \begin{proof}
% \end{proof}
