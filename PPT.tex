\documentclass[9pt]{beamer}
%\include{darkbeamerthemes}
\usepackage{ragged2e}
\def\Z{\ensuremath\mathbb{Z}}
\def\C{\ensuremath\mathbb{C}}
\def\Q{\ensuremath\mathbb{Q}}
\def\N{\ensuremath\mathbb{N}}
\def\R{\ensuremath\mathbb{R}}
\def\M{\ensuremath\mathcal{M}}
\def\N{\ensuremath\mathbb{N}}
\setbeamertemplate{footline}[frame number]

\newcounter{saveenumi}
\newcommand{\seti}{\setcounter{saveenumi}{\value{enumi}}}
\newcommand{\conti}{\setcounter{enumi}{\value{saveenumi}}}
\newcommand{\fraka}{\ensuremath{\mathfrak{a}}}
\newcommand{\frakb}{\ensuremath{\mathfrak{b}}}
\usetheme{PaloAlto}
\usecolortheme{whale}
%\usecolortheme{cormorant}
\title{Algebraic Number Theory}
\author{Rahul Dintyala}
\date{\today}

\begin{document}

\begin{frame}
\titlepage
\end{frame}

\section{Introduction}

\begin{frame}
  \begin{block}{Theorem}
    Let $H$ be a discrete subgroup of $\R^n$. Then $H$ is generated (as a $\Z$-module) by $r$ vectors which are linearly independent over $\R$ (so $r\leq n$).
  \end{block}
\end{frame}

\begin{frame}
  \begin{block}{Theorem}
    Let $G$ be a group and $\mathbb{K}$ a field. Then, distinct characters are linearly independent over $\mathbb{K}$.
  \end{block}
  \textit{Proof}.
  Suppose
  \[a_1 \chi_1 + \ldots + a_n \chi_n = 0\tag{$\star$}\]
  with $a_i \in \mathbb{K}$ not all zero and $n$ minimal with this property. Then ofcourse $n\geq 2$ and $a_i \neq 0$ for all $i$. Since $\chi_1$ and $\chi_2$ are distinct, there exists $h\in G$ such that $\chi_1(h)\neq \chi_2(h)$. Then, for any $g\in G$,
  \[0 = a_1\chi_1(hg)+\cdots+a_n \chi_n(hg) = a_1 \chi_1(h)\chi_1(g)+\cdots+a_n \chi_n(h)\chi_n(g)\]
  which means that $a_1 \chi_1(h)\chi_1+\cdots+a_n \chi_n(h)\chi_n = 0$. Dividing this last expression by $\chi_1(h)$ and subtracting it from $(\star)$ we get :
  \[\left(a_2 - a_2 \frac{\chi_2(h)}{\chi_1(h)}\right)\chi_2 + \cdots + \left(a_n - a_n \frac{\chi_n(h)}{\chi_1(h)}\right)\chi_n = 0 \]
  contradicting the minimality of $n$. Thus, any collection of distinct characters must be linearly independent. \hfill $\blacksquare$
\end{frame}

\begin{frame}
  \begin{block}{Corollary}
    Suppose that $L/\mathbb{K}$ is a finite normal extension of fields and $\sigma_1,\ldots, \sigma_n$ be the distinct automorphisms of $L$. Then these are linearly independent over $L$.
  \end{block}

  \textit{Proof}. Follows from previous theorem by viewing the automorphisms as homomorphisms from $L^*\to L^*$. \hfill $\blacksquare$

  \begin{block}{Corollary}
    Let $L/\mathbb{K}$ be a finite Galois extension of fields of degree $n$. Suppose that $x_1,\ldots, x_n$ is a basis of $L$ over $\mathbb{K}$ and let $\sigma_1, \ldots, \sigma_n$ be the distinct $\mathbb{K}$-automorphisms of $L$. Then, $\det(\sigma_j(x_i)) \neq 0$.
  \end{block}
  \textit{Proof}. Suppose that this determinant is actually zero. Then, there exist $a_1,\ldots, a_n$ not all zero such that $\sum_{j}a_j\sigma_j(x_i) = 0$ for all $1\leq i\leq n$.
  Now, since $x_1,\ldots, x_n$ make a basis of $L$, for any $l\in L$, $\sum_{j}a_j \sigma_j(l) = 0$. Thus, $\sigma_j a_j \sigma_j = 0$ contradicting the previous corrolary. \hfill $\blacksquare$

\end{frame}

\begin{frame}
  \begin{block}{Definition}
    A discrete subgroup of rank $n$ of $\R^n$ is called a lattice in $\R^n$.
  \end{block}
\end{frame}

\begin{frame}
  \begin{block}{Lemma}
    The volume $\mu(P_e)$ is independent of the base chosen for $H$.
  \end{block}
\end{frame}

\begin{frame}
  \begin{block}{Theorem}
    Let $H$ be a lattice in $\R^n$ and let $S$ be a measurable subset of $\R^n$ such that $\mu(S)> \operatorname{vol}(H)$. Then there exist two distinct points $x,y\in S$ such that $x-y\in H$.
  \end{block}
\end{frame}

\begin{frame}
  \begin{block}{Corollary}
    Let $H$ be a full-dimensional lattice in $\R^n$ and let $C\subseteq \R^n$ be a convex set symmetric about the origin $(i.e. x\in C \implies -x\in C)$. Suppose that either :
    \begin{enumerate}
      \item $\vol(C) >  2^n \cdot \vol(H)$, or
      \item $\vol(C) \geq  2^n \cdot \vol(H)$ and $C$ is compact.
    \end{enumerate}
    Then $ \C\cap (H\setminus\{0\})\neq \varnothing$.
  \end{block}
\end{frame}

\begin{frame}
  \begin{block}{Proposition}
    If $M$ is a free $\Z$-submodule of $K$ of rank $n$ and if $(x_i)_{1\leq i\leq n}$ is a $\Z$-base for $M$ then $\sigma(M)$ is a lattice in $\R^n$, whose volume is :
    \[\operatorname{vol}(\sigma(M)) = 2^{-r_2}|\det_{1\leq i,j\leq n}(\sigma_i(x_j))|.\]
  \end{block}

\end{frame}

\begin{frame}
  \begin{block}{Proposition}
    Let $d$ be the absolute discriminant of $K$, let $A$ be the ring of integers in $K$, and let $\mathfrak{a}$ be a non-zero integral ideal of $A$. Then $\sigma(A)$ and $\sigma(\mathfrak{a})$ are lattices. Moreover,
    \[\operatorname{vol}(\sigma(A)) = 2^{-r_2}|d|^{\frac12}\quad\text{and}\quad \operatorname{vol}(\sigma(\fraka)) = 2^{-r_2}|d|^{\frac12}N(\fraka).\]
  \end{block}
\end{frame}

\begin{frame}
  \begin{block}{Proposition}
    Let $K$ be a number field, $n$ its degree, $r_1$ and $r_2$ are integers defined earlier, $d$ the absolute discriminant of $K$, and $\mathfrak{a}$ a non-zero integral ideal of $K$. Then $\fraka$ contains a non-zero element $x$ such that :
    \[|N_{K/\Q}(x)|\leq \left(\frac{4}{\pi}\right)^{r_2}\frac{n!}{n^n}|d|^{\frac12}N(\mathfrak{a}).\]
  \end{block}
\end{frame}

\begin{frame}
  \begin{block}{Corollary}
    With the same notations, every ideal class of $K$ contains an integral ideal $\mathfrak{b}$ such that :
    \[N(\mathfrak{b}) \leq \left(\frac{4}{\pi}\right)^{r_2}\frac{n!}{n^n}|d|^{\frac12}.\]
  \end{block}
\end{frame}

\begin{frame}
  \begin{block}{Theorem}
    For any number field $K$ the ideal class group is finite.
  \end{block}
\end{frame}
\end{document}
